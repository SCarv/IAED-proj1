\documentclass[12pt,a4paper,notitlepage]{article}
\usepackage[portuguese]{babel}
\usepackage[margin=3cm]{geometry}
\usepackage[utf8]{inputenc}
\usepackage[T1]{fontenc}
\usepackage{secdot}
\usepackage{pgfplots}
\usepackage[noend,linesnumbered,ruled]{algorithm2e}
\renewcommand{\O}[1]{$\mathcal{O}(#1)$}
\SetKwFunction{findver}{Find-Fundamental-Vertices}
\SetKwProg{Function}{function}{}{}
\SetKwInOut{Output}{Output}
\SetKwFunction{init}{initialize-graph}
\SetKwComment{Comment}{$\triangleright$\ }{}
\pgfplotsset{compat=1.12}
\begin{document}
\title{\textbf{Análise e Síntese de Algoritmos} \\\large 1º Projeto}
\date{}
\author{Tomás Cunha, nº 81201, Grupo 15}
\maketitle
\section{Introdução}
Este trabalho tem como objetivo desenvolver um algoritmo que permita descobrir as pessoas fundamentais de uma rede.
Uma pessoa é considerada fundamental se o único caminho para partilha de informação entre outras duas pessoas passa obrigatoriamente por essa pessoa.
Este problema pode ser reduzido a encontrar os vértices de corte de um grafo conexo não dirigido, em que os vértices correspondem às pessoas e as arestas às ligações entre as mesmas, sendo semelhante ao problema 22--2 do livro \emph{Introduction to Algorithms}\cite[p.~622]{algs3ed}, em cujos exercícios me baseei para desenvolver a solução.

\section{Descrição da Solução}
O algoritmo utilizado na solução deste problema é uma variação do algoritmo DFS estudado na aula, usando também a noção de \emph{low} que é usada no algoritmo de Tarjan para encontrar componentes fortemente ligadas.
Este \emph{low} é usado para encontrar o vértice menos profundo no grafo ao qual é possível chegar a partir de um dado vértice, permitindo encontrar arcos para trás.
O grafo é representado por listas de adjacências.

A solução, cuja justificação teórica será dada na secção seguinte, pode ser representada em pseudocódigo da seguinte forma:
\\
\\
\begin{algorithm}[H]

		\Function{\init{G}}{
			\ForEach{v $\in$ V[G]}{
					d[v] $\leftarrow$ $\infty$\;
					$\Pi$ [v] $\leftarrow$ NIL\;
					low[v] $\leftarrow$ $\infty$\;
					
			}
			fundamental-count $\leftarrow$ 0\;
			min-element $\leftarrow$ $+\infty$\;
			max-element $\leftarrow$ $-\infty$\;
			time $\leftarrow$ 0;
}

		\caption{Inicializar as estruturas necessárias}
\end{algorithm}

\begin{algorithm}[H]
		\Function{\findver{u}}{
			d[u]$\leftarrow$ time\;
			low[u] $\leftarrow$ time\;
			time $\leftarrow$ time + 1\;
			is-fundamental $\leftarrow$ false\;
			child-count $\leftarrow$ 0\;
			\ForEach{v $\in$ Adj[u]}{
					\uIf{ d[v] = $\infty$ }{
							$\Pi$[v] $\leftarrow$ u\;
							child-count $\leftarrow$ child-count + 1\;
							\findver{v}\;
					\If{$Low[v] \geq d[u]$}{
							is-fundamental $\leftarrow$ true\;
							}
							Low[u] $\leftarrow$ MIN$(Low[u],Low[v])$\;
					}
					\ElseIf{v $\neq$ Parent[u]}{ 
							Low[u] $\leftarrow$ MIN$(Low[u], d[v])$\Comment*{Possível arco para trás}
				}
			}
			\If{$(Parent[u] \neq NIL \:\textbf{and}\: $is-fundamental$)$ \textbf{or} $(Parent[u] = NIL \:\textbf{and}\: $child-count$\ > 1)$}{
					fundamental-count $\leftarrow$ fundamental-count + 1\;
					min-element $\leftarrow$ MIN$($min-element, v$)$\;
					max-element $\leftarrow$ MAX$($max-element, v$)$\;
			}
}
	
	\caption{Encontrar os vértices de corte}
\end{algorithm}

\begin{algorithm}[H]
		Let $G =$ grafo formado a partir do input\;
		\init{G}\;
		\findver{1}\;
		\Output{fundamental-count, min-fundamental, max-fundamental}
\caption{Função principal}
\end{algorithm}
\pagebreak

\section{Análise Teórica}
Os dados do enunciado garantem que haverá sempre um caminho entre qualquer par de pessoas, o que indica que a floresta obtida após uma DFS será composta por uma única árvore.
Há então dois tipos possíveis de vértices que é necessário avaliar: a raiz da árvore, e os restantes vértices.

No caso de o vértice ser a raiz da árvore, será um vértice de corte se e só se tiver pelo menos dois descendentes.
É fácil provar que isto se sucede: se tiver apenas um descendente, uma vez que é a raiz da árvore, não poderá ser um vértice de corte, pois não tem antecessores aos quais o seu antecessor já não estaria ligado.
Se tiver dois ou mais descendentes, uma vez que a procura é em profundidade primeiro, isso significa que, após a recursão terminar para o primeiro filho (ou, no caso de ter \emph{n} filhos, para os primeiros \emph{n$-$1} filhos), o descendente restante não era adjacente a nenhum outro vértice da árvore. Isto significa que, ao remover a raiz da árvore, haverá uma divisão da árvore em duas (ou mais) subárvores. 

No caso de ser outro tipo de vértice, \emph{v} será um vértice de corte se e só se tiver um descendente \emph{w} tal que nem \emph{w} nem os seus descendentes têm um arco para trás para um antecessor do vértice \emph{v}. 
A prova desta afirmação é também simples.
Se \emph{w} não tiver um arco para trás que o ligue a um antecessor de \emph{v}, e nenhum dos seus descendentes tiver esta ligação, a remoção de \emph{v} irá cortar a única ligação de \emph{w} e dos seus descendentes à restante árvore, que será composta por pelo menos um vértice, uma vez que \emph{v} não é a raiz da árvore.
Se o contrário se suceder, ou seja, ou \emph{w} ou um dos seus descendentes tem um arco para trás, então mesmo que \emph{v} seja removido haverá uma ligação entre os antecessores de \emph{v} e os seus descendentes, ou seja, a remoção de \emph{v} não irá ter um efeito no número de componentes ligadas, logo \emph{v} não será um vértice de corte.

Estas duas observações são suficientes para chegar à solução do problema: basta guardar o \emph{lowpoint} de cada vértice ao longo da visita, ou seja, a profundidade mais baixa à qual os descendentes de um dado vértice \emph{v} estão ligados, e conseguimos saber se estes têm ou não um arco para trás.
Para encontrar o vértice raiz da árvore, basta ver qual o vértice que não tem predecessor.

A complexidade da função \textbf{Initialize-Graph} é $\Theta(|V|)$ uma vez que percorre todos os vértices do grafo uma única vez.
Analisando as linhas do \textbf{Algorithm 2} da secção anterior, vemos que as linhas 2--6 e 17--20 são realizadas em $\Theta(1)$, havendo um ciclo nas linhas 7--16 que é executado \emph{E} vezes, em que \emph{E} representa o número de arestas do vértice.
A linha 10 chamará a função recursivamente para todos os vértices do grafo, uma única vez.
Como cada chamada à função percorre todas as arestas de cada vértice, e é chamada uma única vez para cada vértice do grafo, a complexidade total da função, e consequentemente do algoritmo, será $\Theta(|V|+|E|)$, como seria expectável visto que se trata de uma variação da DFS habitual.
\pagebreak

\section{Avaliação Experimental dos Resultados}

Foram realizados 100 testes, aplicando o algoritmo a grafos com entre 20\,000 e 2\,000\,000 Vértices$+$Arestas.
Após desenhar um gráfico com os resultados da medição do tempo de execução desses testes em função da dimensão do grafo, é possível observar que é próximo do gráfico  de uma função do tipo $y=mx+b$, sendo coerente com a complexidade assimptótica $\Theta(|V|+|E|)$ obtida na análise teórica do algoritmo.

\begin{center}
\begin{tikzpicture}
		\begin{axis}[
				title=Relação Tempo-Dimensão,
				xlabel={Dimensão (V+E)},
				ylabel={Tempo (s)}
		]
		\addplot table {test_results.dat};
		\end{axis}
\end{tikzpicture}
\end{center}

\begin{thebibliography}{9}
		\bibitem{algs3ed}
				Thomas H. Cormen,
				Charles E. Leiserson,
				Ronald L. Rivest,
				Clifford Stein,
				\emph{Introduction to Algorithms},
				3rd Edition,
				September 2009
\end{thebibliography}
\end{document}
